\documentclass[12pt,a4paper,TimesNewRoman]{article}
\usepackage{graphicx}
\usepackage[14pt]{extsizes}
\usepackage{cmap}
\usepackage[T2A]{fontenc}
\usepackage[utf8]{inputenc}
\usepackage[english,russian]{babel}
\usepackage[left=2.5cm,right=1.5cm,top=2cm,bottom=2cm]{geometry}
\linespread{1.3} %междустрочный интервал 1.5
%абзацный отступ 1.25??
%выравнивание по ширине??
\usepackage{indentfirst}
\usepackage{graphicx}
\usepackage{subcaption}
\usepackage{mathtools}
\usepackage{amsmath}
\usepackage{amssymb}
\usepackage{amsfonts}
\usepackage{float}
%\usepackage{caption}
\usepackage[unicode,pdftex]{hyperref}
\usepackage{polynom}
\usepackage{multicol}
\setcounter{page}{2}
\usepackage{enumitem}
\usepackage{listings}

\usepackage[unicode, pdftex]{hyperref}
\hypersetup{
    colorlinks=true,    % Включает цветные ссылки вместо рамок
    linkcolor=black,    % Цвет внутренних ссылок (например, на разделы)
    urlcolor=black,     % Цвет URL-ссылок
    citecolor=black,    % Цвет ссылок на литературу
    pdfborder={0 0 0},  % Убирает рамки вокруг ссылок
    pdfstartview=FitH,  % Опция для правильного отображения при открытии
    bookmarksnumbered=true, % Нумерованные закладки в PDF
    bookmarksopen=true      % Раскрывать закладки по умолчанию
}

\lstset{
    language=Verilog,           % Язык - Verilog
    basicstyle=\ttfamily\small, % Шрифт
    keywordstyle=\color{blue},  % Цвет ключевых слов
    commentstyle=\color{black}, % Цвет комментариев
    numbers=left,               % Нумерация строк слева
    numberstyle=\color{black},
    frame=none,               % Рамка вокруг кода
    breaklines=true,            % Перенос длинных строк
    tabsize=4,                  % Размер табуляции
    showstringspaces=false      % Не показывать пробелы в строках
}


\usepackage{caption}
\DeclareCaptionFormat{custom}{#1#2~--- #3} % Формат: "Рисунок 1 — Текст"
\captionsetup[figure]{
    format=custom,
    name=Рисунок,
    labelsep=space, % Убирает двоеточие
}

\begin{document}

\section{Этап 1. Подготовительный}

\paragraph{Неделя 1.} Изучение предметной области, конспект по DDS, АЦП. Анализ существующих решений. Определение ТЗ устройства.
\paragraph{Неделя 2.} Подготовка инструментов для разработки.

\section{Этап 2. Проектирование архитектуры}

\paragraph{Неделя 3.} Разработка системной архитектуры (блок-схема системы). Выбор интерфейсов, проектирование протокола связи. Определение форматов данных. Разделение на модули.

\section{Этап 3. Реализация на SystemVerilog}

\paragraph{Неделя 4.} Реализация базовых модулей (\texttt{sine\_lut, phase\_accumulator, dds\_core}).
\paragraph{Неделя 5.} Память и интерфейсы (\texttt{arb\_wave\_bram, uart\_receiver, protocol\_parser}).
\paragraph{Неделя 6.} Интеграция и управление. (\texttt{control\_interface, dac\_interface, top}).
\paragraph{Неделя 7.} Оптимизация и отладка.

\section{Этап 4. Программирование ESP32}

\paragraph{Неделя 8.} Базовый функционал. UART или SPI драйвер. Протокол передачи. Веб-сервер.
\paragraph{Неделя 9.} Загрузка файла произвольной формы.

\section{Этап 5. Испытания и измерения}

\paragraph{Неделя 10.} Функциональные тесты. Проверка всех форм волн. Диапазон частот. Загрузка произвольной формы.
\paragraph{Неделя 11.} Параметрические измерения. Нелинейные искажения. Фазовый шум. Точность амплитуды.


\end{document}