\documentclass[12pt,a4paper,TimesNewRoman]{article}
\usepackage{graphicx}
\usepackage[14pt]{extsizes}
\usepackage{cmap}
\usepackage[T2A]{fontenc}
\usepackage[utf8]{inputenc}
\usepackage[english,russian]{babel}
\usepackage[left=2.5cm,right=1.5cm,top=2cm,bottom=2cm]{geometry}
\linespread{1.3} %междустрочный интервал 1.5
%абзацный отступ 1.25??
%выравнивание по ширине??
\usepackage{indentfirst}
\usepackage{graphicx}
\usepackage{subcaption}
\usepackage{mathtools}
\usepackage{amsmath}
\usepackage{amssymb}
\usepackage{amsfonts}
\usepackage{float}
%\usepackage{caption}
\usepackage[unicode,pdftex]{hyperref}
\usepackage{polynom}
\usepackage{multicol}
\setcounter{page}{2}
\usepackage{enumitem}
\usepackage{listings}

\usepackage[unicode, pdftex]{hyperref}
\hypersetup{
    colorlinks=true,    % Включает цветные ссылки вместо рамок
    linkcolor=black,    % Цвет внутренних ссылок (например, на разделы)
    urlcolor=black,     % Цвет URL-ссылок
    citecolor=black,    % Цвет ссылок на литературу
    pdfborder={0 0 0},  % Убирает рамки вокруг ссылок
    pdfstartview=FitH,  % Опция для правильного отображения при открытии
    bookmarksnumbered=true, % Нумерованные закладки в PDF
    bookmarksopen=true      % Раскрывать закладки по умолчанию
}

\lstset{
    language=Verilog,           % Язык - Verilog
    basicstyle=\ttfamily\small, % Шрифт
    keywordstyle=\color{blue},  % Цвет ключевых слов
    commentstyle=\color{black}, % Цвет комментариев
    numbers=left,               % Нумерация строк слева
    numberstyle=\color{black},
    frame=none,               % Рамка вокруг кода
    breaklines=true,            % Перенос длинных строк
    tabsize=4,                  % Размер табуляции
    showstringspaces=false      % Не показывать пробелы в строках
}


\usepackage{caption}
\DeclareCaptionFormat{custom}{#1#2~--- #3} % Формат: "Рисунок 1 — Текст"
\captionsetup[figure]{
    format=custom,
    name=Рисунок,
    labelsep=space, % Убирает двоеточие
}

\title{вкр дневник}
\author{Uladzislau Kaplan}
\date{December 2025}

\begin{document}
% -----СОДЕРЖАНИЕ-----
%\thispagestyle{empty}
%\tableofcontents
%\newpage
% --------------------

\section{Введение}

\paragraph{План:} \href{run:./plan.pdf}{ссылка на план выполнения}

\paragraph{Тема:} Генератор сигналов специальной формы на основе схем с программируемой логикой.

\paragraph{Цель:} Разработка генератора сигналов специальной формы на основе схем с программируемой логикой.

\paragraph{Задачи:}
\begin{enumerate}
    \item Анализ технических характеристик генераторов специальной формы (ГССФ) и уточнение требований к его реализации на основе ПЛИС.
    \item Анализ системы формирования частот в базисе ПЛИС.
    \item Разработка кода поведенческого описания ГССФ формы на основе схем с программируемой логикой.
    \item Моделирование ГССФ инструментами САПР Vivado.
    \item Тестовые испытания ГССФ на основе отладочной платы с ПЛИС. Демонстрация воспроизведения сигналов произвольной формы.
\end{enumerate}

\paragraph{Функционал устройства:}
\begin{enumerate}
    \item Предустановленные формы сигналов: синусоида, меандр, пилообразный и треугольный.
    \item Управление основными параметрами: частота, амплитуда, смещение, фаза.
    \item Функицонал "произвольной формы". Загрузка пользовательской формы из внешнего источника.
    \item Опционально добавить возможность сгенерировать конечное количество периодов.
\end{enumerate}

\section{Генерация сигналов}

В устройтсве будет использоваться DDS (Direct Digital Synthesis).

Простейший DDS выглядит так: двоичный счетчик формирует адрес для ПЗУ, куда записана таблица одного периода функции синусоиды, отсчеты с выхода ПЗУ поступают на ЦАП, который формирует на выходе синусоидальный сигнал, подвергающийся фильтрации в ФНЧ и поступающий на выход. 

Для перестройки выходной частоты используется делитель с переменным коэффициентом деления, на вход которого поступает тактовый сигнал с опорного генератора.

\begin{figure}[H]
    \centering
    \includegraphics[width=0.75\linewidth]{/home/buba/Documents/vkr/text/assets/DDS_synth(1).png}
    \caption{Архитектура DDS генератора.}
    \label{fig:dds-arch}
\end{figure}

Частота, амплитуда и фаза сигнала в любой момент времени точно известны и подконтрольны. Единственным элементом, который обладает свойственной аналоговым схемам нестабильностью, является ЦАП. Высокие технические характеристики стали причиной того, что в последнее время DDS вытесняют обычные аналоговые синтезаторы частот. 

Архитектура DDS в моем устройстве будет следующей:

\(\text{CLK GEN} \Rightarrow \text{Аккумулятор фазы} \Rightarrow \text{ЦАП} \Rightarrow \text{Аналоговый сигнал}\)

\subsection{Аккумулятор фазы}

\begin{lstlisting}
logic [31:0] phase_acc;

always_ff @(posedge clk) begin
    phase_acc <= phase_acc + phase_inc;
end
\end{lstlisting}

\subsection{Генератор частоты DDS}
Формула частоты DDS:
\begin{equation}
    F_{out} = (phase\_inc * F_{clk}) / 2^N
\end{equation}
где \(F_{out}\) -- выходная частота, \(F_{clk}\) -- тактовая частота ПЛИС, \(phase\_inc\) -- слово настройки, \(N\) -- разрядность аккумулятора.

Если тактовая частота ПЛИС 100 МГц, разрядность аккумулятора 32 бит, я хочу синусоиду \(F_{out}=1\)кГц, тогда:
\[phase\_inc = F_{out} * (2^N) / F_{clk}\]
\[phase\_inc = 1000 * 4294967296 / 100000000 \approx 42949673 (0x028F5C29 \text{ в hex})\]

Аккумулятор фазы -- это просто счетчик, который на каждом такте увеличивается на \texttt{phase\_inc}

\subsubsection{Режимы работы DDS в ГССФ}

\paragraph{Синусоида} Для генерации синусоиды для отсчетов из таблицы брать старшие биты аккумулятора фазы.
\[phase\_acc \Rightarrow \text{страшие биты} \Rightarrow sine\_table[addr] \Rightarrow \text{ЦАП}\]

\paragraph{Меандр} Для генерации меандра заданной частоты нужен только аккумулятор фазы, из которого буду брать старший бит.
\begin{lstlisting}
square_wave = phase_acc[31]; // 0 or 1
\end{lstlisting}

Когда \texttt{phase\_acc} переполняется, старший бит меняется \(0-1-0-1\dots\) 

\paragraph{Пилообразный сигнал} Для генерации пилообразного сигнала можно 

взять старшие 12 бит фазы как амплитуду.
\begin{lstlisting}
sawtooth = phase_acc[31:20];
\end{lstlisting}
Получится \(0,1,2,\dots,4095,0,1,2\dots\) что и будет являться пилой.

\paragraph{Треугольный сигнал} Для его генерации возьмем старший бит аккумулятора фазы как индикатор возрастания или убывания.

\begin{lstlisting}
if (phase_acc[31] == 0)
    triangle = phase_acc[30:19]; // increment
else
    triangle = ~phase_acc[30:19]; // decrement
\end{lstlisting}

\paragraph{Произвольная форма} Вместо таблицы \texttt{sine\_table} используется BRAM с пользовательскими данными.
\begin{lstlisting}
arb_wave = user_bram[phase_acc[31:22]];
\end{lstlisting}

\subsection{Проблемы DDS и их решения}

\subsubsection{Ступенчатость на выходе ЦАП}
Решение: Фильтр низких частот (аналоговый RC/LC фильтр)

\subsubsection{Спектральная чистота}
Решение: 
\begin{enumerate}
    \item Увеличение разрядности таблицы
    \item Добавление dithering
    \item Интерполяция между точками таблицы
\end{enumerate}

\section{Загрузка сигнала произвольной формы}

Можно использовать ESP32 микроконтроллер для загрузки файла с отсчетами через веб-интерфейс.

Связать ESP32 - ПЛИС можно через UART или SPI. SPI будет работать быстрее.

Возможно придется определить протокол обмена:
\begin{lstlisting}
    SET_FREQ:<32-bit-phase-word>\n
    SET_WAVE:SINE\n
    SET_WAVE:ARB\n
    LOARD_ARV:<data-size>\n
\end{lstlisting}

ESP32 будет либо создавать точку доступа WiFi для подключения к нему и хостить сервер в локальной сети, либо будет подключатся к WiFi и хостить сервер там, но я думаю лучше будет просто создавать точку доступа WiFi.

%\newpage
%\begin{thebibliography}{99}
%    \bibitem{}
%\end{thebibliography}

\end{document}
