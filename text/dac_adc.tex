\documentclass[12pt,a4paper,TimesNewRoman]{article}
\usepackage{graphicx}
\usepackage[14pt]{extsizes}
\usepackage{cmap}
\usepackage[T2A]{fontenc}
\usepackage[utf8]{inputenc}
\usepackage[english,russian]{babel}
\usepackage[left=2.5cm,right=1.5cm,top=2cm,bottom=2cm]{geometry}
\linespread{1.3} %междустрочный интервал 1.5
%абзацный отступ 1.25??
%выравнивание по ширине??
\usepackage{indentfirst}
\usepackage{graphicx}
\usepackage{subcaption}
\usepackage{mathtools}
\usepackage{amsmath}
\usepackage{amssymb}
\usepackage{amsfonts}
\usepackage{float}
%\usepackage{caption}
\usepackage[unicode,pdftex]{hyperref}
\usepackage{polynom}
\usepackage{multicol}
\setcounter{page}{2}
\usepackage{enumitem}
\usepackage{listings}

\usepackage[unicode, pdftex]{hyperref}
\hypersetup{
    colorlinks=true,    % Включает цветные ссылки вместо рамок
    linkcolor=black,    % Цвет внутренних ссылок (например, на разделы)
    urlcolor=black,     % Цвет URL-ссылок
    citecolor=black,    % Цвет ссылок на литературу
    pdfborder={0 0 0},  % Убирает рамки вокруг ссылок
    pdfstartview=FitH,  % Опция для правильного отображения при открытии
    bookmarksnumbered=true, % Нумерованные закладки в PDF
    bookmarksopen=true      % Раскрывать закладки по умолчанию
}

\lstset{
    language=Verilog,           % Язык - Verilog
    basicstyle=\ttfamily\small, % Шрифт
    keywordstyle=\color{blue},  % Цвет ключевых слов
    commentstyle=\color{black}, % Цвет комментариев
    numbers=left,               % Нумерация строк слева
    numberstyle=\color{black},
    frame=none,               % Рамка вокруг кода
    breaklines=true,            % Перенос длинных строк
    tabsize=4,                  % Размер табуляции
    showstringspaces=false      % Не показывать пробелы в строках
}


\usepackage{caption}
\DeclareCaptionFormat{custom}{#1#2~--- #3} % Формат: "Рисунок 1 — Текст"
\captionsetup[figure]{
    format=custom,
    name=Рисунок,
    labelsep=space, % Убирает двоеточие
}

\begin{document}

\section{Дискретизация}

Дискретизация -- это процесс преобразования непрерывного аналогового сигнала в последовательность дискретных значений, снимаемых через равные промежутки времени, чтобы его можно было обрабатывать и хранить в цифровом виде, а ключевой параметр -- частота дискретизации, определяющее количество отсчетов в секунду.

\paragraph{Как выполняется дискретизация} Сигнал обладает определенными характеристиками:
\begin{itemize}
    \item Частота сигнала
    \item Амплитуда
    \item Длительность сигнала
\end{itemize}

Для выполнения дискретизации необходимо знать частоту дискретизации.
\paragraph{Теорема Котельникова (Найквиста-Шеннона)} Теорема утверждает, что аналоговый сигнал с финитным спектром (т. е. со спектром, ограниченным некоторой частотой \(f_m\)) полностью определяется последовательностью своих дискретных значений (отсчетов), взятых через интервалы времени \(\Delta t \leq 1/(2f_m)\), то есть с частотой дискретизации \(f_d \geq 2f_m\). Другими словами, при выполнении этого условия аналоговый сигнал можно точно восстановить по его дискретным значениям.

\begin{figure}[H]
    \centering
    \includegraphics[width=1\linewidth]{assets/discretization.png}
    \caption{Дискретизация аналогового сигнала и быстрое преобразование Фурье дискретного сигнала. Частота сигнала 5Гц, частота дискретизации 50Гц}
    \label{fig:discretization1}
\end{figure}

\subsection{Алиасинг}

В обработке сигналов аласингом называют эффект, приводящий к наложению парциальных спектров аналогового сигнала после его дискретизации при невыполнении условий теоремы Котельникова.

\begin{figure}[H]
    \centering
    \includegraphics[width=1\linewidth]{assets/aliasing.png}
    \caption{Явление алиасинга (наложения спектров) при невыполнении теоремы Котельникова.}
\end{figure}

Пример дискретизации модели аудиосингала (сложение трех нот и наложение шума).
\begin{figure}[H]
    \centering
    \includegraphics[width=1\linewidth]{assets/discrete_example.png}
    \caption{Пример дискретизации аудиосигнала с разной частотой дискретизации}
\end{figure}

\section{Квантование}

В обработке сигналов -- разбиение диапазона отсчетных значений сигнала на конечное число уровней и округление этих значений до одного из двух ближайших к ним уровней. При этом значение сигнала может округляться либо до ближайшего уровня, либо до меньшего или большего из уровней в зависимости от способа кодирования.

\begin{figure}[H]
    \centering
    \includegraphics[width=1\linewidth]{assets/quantization.png}
    \caption{Квантование сигнала с разрешением в 3 бита.}
\end{figure}

Для оценки ошибки квантования используют параметр, являющийся отношением сигнала к шуму (Signal-to-quantization-noise ratio, \(SQNR\)). Измеряется в децибелах. Чем больше это отношение тем меньше будет ошибка квантования.

Существуют различные типы квантования.
\begin{figure}[H]
    \centering
    \includegraphics[width=1.0\linewidth]{assets/types_of_quantization.png}
    \caption{Различные типы квантования и SQNR}
\end{figure}

\section{Аналогово-цифровой преобразователь, АЦП}

АЦП преобразует аналоговый сигнал в цифровой, то есть сигнал проходит два преобразования: дискретизацию и квантование. Двумя ключевыми характеристиками АЦП являются разрядность (от нее зависит количество уровней квантования амплитуды) и частота дискретизации.

Количество уровней квантования:
\[n=2^N, \text{\(N\)-- разрядность АЦП}\]

\begin{figure}[H]
    \centering
    \includegraphics[width=1.0\linewidth]{assets/dac.png}
    \caption{Процесс дискретизации и квантования в АЦП с разрядностью 8 бит и частотой дискретизации 1кГц}
\end{figure}

\begin{figure}[H]
    \centering
    \includegraphics[width=1.0\linewidth]{assets/sqnr_bit_correlation.png}
    \caption{Квантованый сигнал в зависимости от разрядности АЦП}
\end{figure}

\begin{figure}[H]
    \centering
    \includegraphics[width=1.0\linewidth]{assets/sqnr_bit_func.png}
    \caption{Зависимость SQNR от разрядности АЦП}
\end{figure}

\section{Цифро-аналоговый преобразователь, ЦАП}

ЦАП -- устройство для преобразования цифрового кода в аналоговый сигнал. Цифро-аналоговые преобразователи вляются интерфейсом между дискретным цифровым миром и аналоговыми сигналами.

ЦАП характеризуется двумя ключевыми характеристиками: разрядностью и напряжением питания. Выходной сигнал без обработки является ступенчатым, что не будет допустимым результатом в большинстве устройств. Для сглаживания ступеней используется интерполяция и фильтр низких частот (ФНЧ).

\begin{figure}[H]
    \centering
    \includegraphics[width=1.0\linewidth]{assets/dac_with_interpolation.png}
    \caption{Восстановление аналогового сигнала из цифрового}
\end{figure}

\end{document}